\section{海流}
\subsection{海流概述}
	\subsubsection{概述}
	\begin{enumerate}[1)]
		\item 海流是大规模相对稳定的流动
		\item 海流是海水重要的普遍运动形式之一,是海洋环境中重要的物理现象
		\item 所有的海洋现象都与海水运动(环流)有关
		\item 将大洋各部分联系起来,使大洋的各种水文、化学要素及热盐状况保持长期相对稳定。
		\item 海流首尾相连 环流
	\end{enumerate}
	
	\subsubsection{海流的影响}
		\begin{enumerate} [1)]
			\item 海流的大小、方向与分布关系到海洋工程结构物的手里与稳定性能
			\item 海底泥沙的冲淤与迁移规律
			\item 海洋气候变化问题
			\item 海洋污染灾害
			\item 海洋渔业捕捞和海水养殖业的发展
		\end{enumerate}
	\subsubsection{海流命名方法}
		\begin{forest}
			for tree = {grow'=east}
			[命名
			[起因[风[漂流][风海流]][等压面与等势流不一致[密度流][倾斜流]]]
			[受力
			[地转流]
			[惯性流]]
			[区域[洋流][陆家流][赤道流][东西边界流]]
			[温度[暖流][寒流]
			]
			]
		\end{forest}
\subsection*{海流运动方程}
	\subsubsection{综述}
	两大海水作用力:
	\begin{enumerate}[1)]
		\item 引起海水运动的力,如重力,压强梯度力,风应力,引潮力等
		\item 由海水运动后派生出来的力 地转偏向力、摩擦力等
	\end{enumerate}
	\subsubsection{重力}
	g是纬度和深度的函数,重力和静止的海面垂直,静态的海洋表面是等势面
	\subsubsection{压强梯度力、海洋压力场}
	等压面: 海洋中压强相等的点组成的面
	内压场: 等压面的倾斜与海水的密度分布密切相关,密度分布不均匀就必然是等压面发生相对的倾斜。
	外压场: 风、气压变化、江河径流、降水增水等引起海面倾斜所产生的压力场
	\subsubsection{地转偏向力}
	科氏力的方向与运动物体方向垂直,在北半球,水平科氏力指向运动物体的右方,使运动方向不断向右偏
	\subsubsection{切应力}
	 两层流体做相对运动时,由于分子粘滞性,在其界面上产生的一种切向作用力。
	 \subsubsection{引潮力}
	 天体引力与惯性离心力的合力

\subsection{地转流}
地转流:水平压强梯度力与科氏力取得平衡
深海环流可近似看成地转流
$v_g=\frac{g}{f}tg\beta$
f 为科氏参量,$\beta$ 为等压面和等势面的夹角
\subsection{风海流}
水平湍流切应力与科氏力相平衡。
	\subsubsection{深海漂流特性}
	\begin{enumerate} [a)]
		\item 流速随深度z负向深度的加大而按指数型衰减
		\item 流失量与x轴的夹角随深度加深而不断右偏
	\end{enumerate}
	
	\subsubsection{浅海基本特征}
	有限深度海洋,受到海底摩擦作用,浅海风海流与无线深海漂流结构的差异
	水深h越浅,从上层到下层的流速矢量越是趋近风矢量的方向
	\subsubsection{风海流的体积运输}
	无限深海:只在x方向上存在,北半球海水的体积运输方向与风矢量垂直,且指向右方
	浅海:在x、y方向上都存在,运输方向偏离风矢量角度小于九十度,水深越浅偏角越小
	\subsubsection{上升流与下降流}
	风海流的体积运输——海水的集聚——海水堆积——补偿流——海水的垂向运动。
\subsection{大洋环流}
海面风力和热盐效应下,海水从某领域向另一海域流动而形成的首尾相连的独立循环系统或流涡
	\subsubsection{风生大洋环流}
	风应力为驱动力的环流运动
	西向强化,流幅窄,流速大
	科氏力随纬度变化是主要原因
	\subsubsection{热盐环流}
	以温、盐变化产生的密度梯度为驱动力的环流运动
	在大洋中下层占主导地位,流动相对缓慢
	是形成大洋中下层温盐分布特征及海洋层化结构的主要原因
	深海环流近似看成地转流
	
\subsection{海气相互作用和厄尔尼诺现象}
	\subsubsection{海洋在气候系统的地位}
	地球气候系统最重要的组成部分
	海洋-大气相互作用是气候变化问题的核心内容
	海洋在气候系统的重要地位有海洋自身的性质所决定
	海洋对大气系统热力平衡的影响
	海洋对水汽循环的影响
	海洋对大气运动的调谐作用
	海洋对温室效应的缓解作用
	
	\subsubsection{基本特征}
	\begin{enumerate}[a)]
		\item 海洋主要通过向大气输送热量,尤其是提供潜热
		\item 大气主要通过风应力向海洋提供动量,改变洋流及重新分配海洋的热含量
		\item 海洋混合层 海洋相互作用通过大气和海洋混合层间热量、动量和质量直接交换而奏效
	\end{enumerate}
	
	\subsubsection{厄尔尼诺现象}
		亚尼娜现象

\subsection{中国近海环境}
	外海流系及沿岸流系组成
	外海流系包括 黑潮、台湾暖流、对马暖流和黄海暖流
	沿岸流系:低盐性质的江河入海径流与盛行季风引起的风海流
				渤海风海流
				黄海风海流
				东海风海流
				南海沿岸流
	
	
\subsection {海流作用力}
	产生涡激
	危害 构件自振频率与旋涡发放频率相近,造成共振破坏
	措施 刚性加固 用扰流装置
	

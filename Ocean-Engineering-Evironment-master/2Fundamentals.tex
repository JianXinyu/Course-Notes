\section{Fundamentals of Ocean Environment}
	\subsection{Earth}
		大地水准面:地球等势面,地面点高程起算的基准面.大地体.
		
		标准椭球体,	正球体
		\begin{forest}
			for tree = {grow'= east}
			[同心圈层结构
				[外圈
					[大气圈]
					[水圈]
					[生物圈]
					]
				[内圈
					[地壳]
					[地幔]
					[地核]
					]
				]
		\end{forest}
	\subsection{海洋沉积与海岸地貌形态}
	\begin{enumerate}[1)]
		\item 海洋沉积:通过海水搬运而沉降覆盖堆积在海底的泥,砂等无机物质和生物残骸等有机物质的统称.
			\begin{forest}
				[沉积物来源
					[陆源物质]
					[生物残骸]
					[火山物质]
					[宇宙物质]
					[化学沉淀物]
					]
			\end{forest}
		
		\item 海洋地形
			\begin{forest}
				for tree = {grow'=east}
				[海洋地形
					[海岸地形
						[海岸线:变化]
						[海岸带:经济最发达
							[陆上部分]
							[潮间带]
							[水下岸坡]
							]
						]
					[海底地形
						[大陆边缘]
						[大洋底]
						]	
					]
			\end{forest}
		\item 
			\begin{forest}
				[海蚀作用
					[冲蚀作用]
					[磨蚀作用]
					[溶蚀作用]
					]
			\end{forest}
		\item 
			\begin{forest}
				[海岸类型
					[基岩海岸]
					[沙砾质海岸]
					[淤泥质海岸]
					[生物海岸]
				]
			\end{forest}
		\item 
			\begin{forest}
				for tree = {grow'=east}
				[海底形态
					[大陆边缘
						[大陆架\textbf{大陆向海底的自然延伸}上下限]
						[大陆坡陡倾上下限]
						[大陆隆]
						]
					[大洋底
						[大洋盆地]
						[大洋中脊]
						]
					]	
			\end{forest}
			
	\end{enumerate}
	
	\subsection{海水温度及其主要热性质}
		\begin{enumerate}[1)]
			\item \textbf{海水温度}
			\item  
				\begin{forest}
					for tree = {grow'=east}
					[热性质-海水的固有性质
						[\textbf{热容} 3091]
						[\textbf{比热}]
						[\textbf{蒸发热}\underline{对大气和海洋的热状况}影响big 比如 t上升而下降 受s影响小]
						[\textbf{热传导}t升而升 s升而略降]
						[\textbf{体积热膨胀系数}
						[\textbf{绝热变化}]]
						]
				\end{forest}
		\end{enumerate}
	
	\subsection{海水盐度-标度,重要特性,基本参数}
	
	\subsection{海水密度及海水状态方程}
		是海水状态参数温度,盐度,压力与密度或比容之间的函数式
		
	\subsection{大洋温度,盐度,密度的特征}
		\begin{enumerate}
			\item 	海洋热收支
			\textbf{太阳辐射能$Q_S$,海洋有效回辐射$Q_B$,海汽间的热感交换$Q_H$,蒸发$Q_E$(海汽热交换的三种方式)}.
			
			海面热收支余量$Q_t=Q_S-Q_B\pm Q_H\pm Q_E$,大洋$Q_t=0$
			
			海洋内部热交换:垂直方向的热输运$Q_Z$-海面的风浪流引起的涡动混合,水平方向$Q_A$-海流
			
			全热量平衡
			\item 大洋表层的分布特征:
				\begin{enumerate}
					\item 等温线大致沿纬线呈带状分布
					\item 
				\end{enumerate}
			\item 垂直分布
				\begin{enumerate}
					\item 水温大体上随深度的增加呈不均匀递减,沿深度分布存在层化态.
					\textbf{跃层:水文要素在铅直方向上出现的跃变水层}.大洋主温跃层,季节性温跃层.
					
					温跃层\textbf{强度标准}
				\end{enumerate}
			\item 大洋水温的变化:1)日变化很小2)大洋表层温度的年变化主要受制于太阳辐射的年变化3)年变幅因海域不同以及海流性质、盛行风系的年变化和结冰融冰等因素的变化而不同
			
			\item 海洋的水平衡
			\item 盐度的分布变化特征.
			
				盐跃层:盐度垂直梯度较大的水层.永久性,季节性盐跃层. 盐跃层强度标准
			\item 密度的分布变化
		\end{enumerate}
	\subsection{中国近海温度,盐度,密度的特征}
		\begin{enumerate}
			\item 水温:
				\begin{enumerate}
					\item 水平:冬季表面水温自北向南显著增高;2月最低,8月最高
					\item 垂直:显著的季节变化
					\item 近海水温变化:1)日变化与气温日变化、天气状况有关,变化趋势一致。
					2)随深度增加,日变化差减小。
					3)日变化:沿岸海区> > 中央海区,北部海区> > 南部海区。
					4)年变化比日变化大许多,自北向南递减。
					5)浅海、边缘海等受大陆影响比大洋年变幅大
				\end{enumerate}
			\item 盐度:河口低,外海高.北海区低,渤海,南高,南海. 浅海季节性跃层在各海区发达
			\item 密度:1)南部海区低,北部海区高.2)冬季明显大于夏季.3)夏季出现强大的密度跃层
		\end{enumerate}
	
			
			
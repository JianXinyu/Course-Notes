\section{Wind}
	\subsection{summary}
		\begin{enumerate}[1]
			\item 风是空气相对于地面的水平运动,由气压差产生
			\item 风的作用:大气显示能量的一种方式,影响整个地球大气的运动
			\item 复杂性(紊动性与阵发性) 阵风性 $F_a$ 气旋,寒潮->大风->海洋结构物破坏
			\item 风(浪)载荷:海洋工程结构物的\textbf{设计控制载荷}
			\item 风向.来向, 16个方位, 风玫瑰图-风在各个方向的强弱和出现次数
			\item \textbf{蒲福风力等级表}
		\end{enumerate}
	
	\subsection{大气压强场和风场}
		\begin{enumerate}[1]
			\item	\textbf{大气压强}p:观测高度到大气上界单位面积上铅直空气柱的重量.
					\textbf{一个标准大气压}:温度为0 $^\circ$C ,纬度为45度的海平面的大气压.1013.25 hPa
			\item 自由大气:边界层(地表上1-1.5km)以上的理想大气
						\begin{enumerate}
							\item 气压梯度力
							\item 科里奥利力\textbf{变向}
							\item 地转风 \textbf{水平等速直线},自由大气中实际风的良好近似.背风而立,高压在右,低压在左.
							
							地转平衡:自由大气中水平气压梯度力与科里奥利力二者的平衡.
							\item 梯度风,对应气旋效应显著的风场.背风而立,高压在右,低压在左.绕高压中心作顺时针方向运动,绕低压中心作逆时针方
							向运动。
							
							梯度平衡:自由大气中,水平气压梯度力,科里奥利力与等速曲线运动离心率三者平衡.
						\end{enumerate}
			\item 
				\begin{forest}
					for tree = {grow'=east}
					[风系
						[大规模
							[全球性的大气环流{}单圈{}三圈经圈环流模型]
							]
						[中规模
							[季风{}大范围盛行风向{}雨季旱季]
							[台风]
							]
						[小规模
							[海陆风]
							]
						]
				\end{forest}
		\end{enumerate}
	
	\subsection{风速计算}
		\begin{enumerate}
			\item 地转风速$V_g$
			\item 梯度风速$V_{gr}$
			\item 海面风速$V_s$
			\item \textbf{标准高度10m},高度<100m,对数公式;>100m,指数公式
			\item \textbf{10min时距}
			\item 观测站资料:\textbf{风速,风向,最大风速(年最大风速xuan)与极大风速,常向风,强向风. 主导风向. 重现期30,50,100}
		\end{enumerate}
	
	\subsection{风对结构物的作用力}
		\begin{enumerate}
			\item 作用力
			\item 风载荷规范计算:海上移动平台,海上固定平台,API,LR规范
			\item 涡激现象:刚性较低的\textbf{细长构件}
			
			稳定风绕过圆柱->产生交错排列的卡门涡街->产生\textbf{横向力,横向振动,斯特劳哈尔数} 
		\end{enumerate}
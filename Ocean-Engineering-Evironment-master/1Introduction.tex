\section{Introduction}
	\subsection{Defination}
		\begin{enumerate}[a)]
			\item 海洋工程是基于\underline{科学的原理和工程技术方法}对海洋及海洋资源进行\underline{研究、开发、利用与保护}的一项海上工程活动;
	%			\item 利用工程原理来分析、设计、发展和管理一些在水中环境中运行的系统;
	%			\item 对海洋进行研究、利用的一门应用性学科,包括海洋的各种结构物的建造、施工及营运等方面;
	%			\item 指以开发、利用、保护、恢复海洋资源为目的,并且工程主体位于海岸线向海一侧的新建、改建、扩建工程;
	%			\item 应用海洋基础科学和相关的技术学科,从工程角度开发利用海洋,叫做海洋工程。
		\end{enumerate}
\subsection{海洋的基本特性}
海洋是地球表面包围陆地和岛屿的广大而连续的含盐水体.面积广大而相互连通,陆地被海洋环抱与隔离.7:3,北陆南水,3795m.
\begin{enumerate}[1)]
	\item 根据海洋要素特点,地貌及形态特征	
	\begin{forest}
		for tree = {grow'=east}
		[海洋
		[主体[洋]]
		[附属
		[海]
		[海湾]
		[海峡]
		]
		]
	\end{forest}
	\item	\begin{forest}
		for tree = {grow'=east}
		[海水成分g
		[96.5\%水]
		[是复杂的多种组分水溶液,极大的溶解能力]
		[含有无机盐类和溶解气体,有机与无机的悬浮物质]
		[组成恒定]
		]
	\end{forest}
	\item 海洋是全球气候的调节器,太阳辐射是地球的能量来源,其$\frac{4}{5}$被海洋吸收,3100:1
	\item 运动是海洋最基本的特征
	\item 风暴潮,海浪,海啸,海冰
\end{enumerate}

\subsection{海洋科学发展}
\begin{enumerate}[1)]
%	\item 海洋科学:为了探索和研究海洋中各种现象和过程的发生、发展、性质和演变规律,由最初的感性认识、观测、航海探险发展为系统的、综合性的理论研究体系。
%	\item 历史阶段:地理大发现时代->单船考察时代->多船考察时代->国际合作调查研究
%	\item Heroes
%	\item Results
%	\item 如今对海洋科学全面认识和高速发展
%	\item 国际海洋科学组织:世界气象组织,政府间海洋学委员会,海洋研究科学委员会,国际生物海洋学协会,海洋地质学委员会
%	\item g海洋调查手段:
%	海洋调查船,
%	盐度-温度-深度仪,
%	声学多普勒流速剖面仪,
%	海洋浮标,
%	气象卫星,
%	海洋卫星,
%	地层剖面仪,
%	侧扫声纳,
%	水下机器人,
%	\item g海洋科学研究方法
%	依赖于直接观测各种时空尺度分布与变化有计划的、长期的、连续
%	的、系统的、多层次的、区域代表性的分析观测数据,认识海洋中的各种现象和规律。
%	\item g现代海洋调查和探测技术:水声,遥感,深潜,电子,计算机??
%	\item 海洋资源开发技术:
%	海洋生物技术,
%	海水资源综合利用技术,
%	海洋油气资源的勘探与开发技术,
%	深海资源勘探与开发技术,
%	海洋能源开发技术,
%	\item 海洋环境监测、预报和环境保护技术:
%	卫星遥感技术、系留浮标和漂流浮标监测技术、走航拖曳式XBT观测技术等;
%	近海区域建立完整的立体监测系统;
%	数值预报、统计预报和专家系统及短、中、长期预报;
%	海洋环境保护与生态修复技术.
%	\item 现代海洋观测系统: 以卫星遥感为主,辅以航空遥感、调查船调查、锚泊浮标和岸站系统的现代海洋观测。
	\item \textbf{中国近海海洋综合调查与评价项目tian}
		\begin{enumerate}
			\item 范围: 内水、领海和部分领海以外海域
%			\item 手段: 卫星遥感、航空遥感、船载声学探测、岸基探测、拖曳深潜等
%			\item 目的: 为海洋资源开发利用、综合管理、减灾防灾、海防建设以及维护海洋生态环境、推动沿海经济的可持续发展提供科学依据
%			\item 调查项目: (1) 物理海洋、海洋地形地貌、海洋地质和地球物理、海洋生态和生物、海洋化学等重要海洋要素.
%			(2) 海岛、海岸带、海域使用、海洋灾害、再生能源、海水利用以及沿海人文社会基本状况等
%			\item "大洋一号"大洋科考
			\item 南极长城站,中山站. 北极黄河站.承担南、北极考察任务的“雪龙”号极地考察船
			\item <国家中长期科学和技术发展规划纲要>-前沿技术:海洋环境立体监测技术,大洋海底多参数快速探测技术,天然气水合物开发技术,深海作业技术.
		\end{enumerate}
\end{enumerate}
%	\subsection{海洋资源}
%	\begin{forest}
%		for tree = {grow'=east}
%		[海洋资源
%			[生物资源]
%			[海底矿产资源
%				[滨海矿砂]
%				[石油天然气]
%				[多金属结合]
%				[富钴结壳]
%				[天然气水合物]
%				[海底热液硫化物]
%				[磷钙石和海绿石]
%				]
%			[海水资源]
%			[海洋能源]
%			[海洋空间资源]
%			[海洋旅游资源]
%			]
%	\end{forest}
	
	\subsection{中国沿岸近海海域特点}
		\begin{enumerate}[1)]
			\item 特点:东南两面濒海,范围面积广300W,海岸线长,岛屿众多
			\item 渤海面积最小,浅海,冬天沿岸大都冰冻(tian).黄海浅海.东海兼具浅海和深海特征,生产力最高.南海深海.
					盐度:渤海<黄海<东南
			\item 海洋法
			
			日内瓦会议->临海与毗连区公约,公海公约,捕鱼与养护公海生物资源公约,大陆架公约.
			联合国海洋法公约94年生效,正式划分8个海域.
				\begin{enumerate}
					\item 内水:领海\underline{基线}向内一侧的全部水域.
					(\underline{中华人民共和领海及毗连区法}:: “中国大陆及其沿海岛屿的领海以连接大陆岸上和沿海岸外缘岛屿上各基点之间的各直线为基线)
					\item \textbf{12海里领海制度}:沿岸国对其领海,领海的上空及其海床和底上享有主权
					\item \textbf{毗连区}:毗连领海,宽度为从领海基线向外海量起不超过24海里的区域.
					\item \textbf{200海里专属经济区制度和大陆架制度}:沿海国领海以外并邻接领海的一个国家管辖的海域
				\end{enumerate}
			\item 
			
		\end{enumerate}

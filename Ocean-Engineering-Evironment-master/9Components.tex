\section{海洋环境保护}
	\subsection{海水的化学组成}
		海水是含有\textbf{多种盐类}的水溶液,盐分主要来源于地壳岩石风化产物,火山喷出物及河流输送的溶解盐.水平,垂直分布.
		\begin{enumerate}[1)]
			\item 主要溶解成分 5种阳离子+5种阴离子+硼酸分子, 保守元素
			\item 微量元素 \textbf{镉}
			\item 溶解气体-溶解氧,二氧化碳
			\item 营养元素是海洋植物生长必需的营养盐-生物制约元素
			\item 有机物质主要来源于海洋生物的分泌、排泄、分解等产物及其衍生物,大气和河流的陆源有机物质.人造有机化合物的\textbf{危害}:不能被细菌或简单的化学反应分解,滞留时间长,脂溶性,难排除,在食物链中被富集,有高浓度的毒性。
			\item 化学资源:盐,镁溴铀钾,从海水中综合提取各种物质的生产过程就是海水的综合利用	
		\end{enumerate}
	
	\subsection{海洋金属腐蚀与防护}
		金属在环境介质作用下所引起的破坏或变质,不可逆转.化学,电化学腐蚀.
		\begin{enumerate}
			\item 海水腐蚀特点:海水是一种强电解质溶液,有大量溶解氧,$Cl^-$浓度高
			\item \begin{forest}
					for tree= {grow'=east} 
						[海洋腐蚀环境
							[化学因素{} 溶解氧 含盐量 pH值]
							[物理因素{} 流速 潮汐 温度]
							[生物因素{} 代谢物及尸体分解物含有硫化氢等酸性成分 破坏保护涂层]
							]
				\end{forest}
			\item 海洋腐蚀环境区域划分
			\item \textbf{海洋环境腐蚀类型}
				\begin{enumerate}
					\item 均匀腐蚀
					\item 点蚀
					\item 缝隙腐蚀
					\item 冲击腐蚀
					\item 空泡腐蚀
					\item 电偶腐蚀
					\item 腐蚀疲劳
				\end{enumerate}
			\item 金属防腐
				阴极保护,防蚀涂料(屏蔽,防蚀)	
			
		\end{enumerate}
	
	\subsection{海洋生物及其环境}
		海洋污损生物:在船舶及海上建筑物、海中仪器仪表上\textbf{附着繁殖} ,带来严重的\textbf{生物污损}问题的生物。
		危害:
		\begin{enumerate}
			\item 增加了船舶的\textbf{重量}和表面\textbf{粗糙度} ,增加了船舶的航行\textbf{阻力},消耗了更多\textbf{燃料}
			\item 增加了海洋平台桩柱的\textbf{直径}大小,桩柱结构流体动力系数在选用上要综合考虑海洋生物附着厚度的影响。
			\item \textbf{堵塞}船舶平台上的给排水管道等
			\item \textbf{破坏}船舶与海洋结构物的防腐涂层
			\item \textbf{影响}到海中仪器仪表功能的正常使用
		\end{enumerate}
		
	\subsection{海洋生态系统,重要}
		\textbf{生态系统}:一定的空间内\textbf{生物}成分和\textbf{非生物}成分通过物质循环和能量的流动互相作用,互相依存,互相调控而构成的一个生态学功能单位.
		
		特征:有自我调节能力但又限度,能保持自身的生态平衡,克服和消除外来干扰.
		
		\begin{forest}
			for tree = {grow'=east}
			[,phantom
				[非生命部分
				[无机物质]
				[有机化合物]
				[气候因素]
				[特定环境因素]
				]
				[生命部分
					[生产者 {}光合作用 {}物质 能量]
					[消费者]
					[分解者]
					]
				]
		\end{forest}
	
	\subsection{海洋环境问题,重要}
		\begin{enumerate}
			\item 海洋环境污染:污染物质进入海洋,超过海洋的自净能力.
			
			Definition-1982联合国海洋法公约:人类直接或间接把\textbf{物质或能量}引入海洋环境,其中包括河口港湾,以致於造成或可能造成\textbf{损害}生物资源和海洋生物, \textbf{危害}人类健康, \textbf{妨碍}捕鱼和海洋的其他正当用途在内的各种海洋活动,\textbf{损害}海水使用质量和\textbf{伤及}环境美观等\textbf{有害}影响
			
			污染来源:航运业,海水养殖业,工业废弃物,农业化肥和农药,城市生活污水.
			
			赤潮:水体中某些微小的\textbf{浮游藻类,原生动物或细菌},在一定的环境条件下\textbf{突发性}地增殖和聚集而引起一定范围内一
			段时间中\textbf{水体变色}的一种\textbf{有害}的生态现象.
			
			{}颜色:红黄绿褐
			
			{}严重破坏生态平衡,损害海洋生态系统结构与功能.
			
			\begin{forest}
				for tree = {grow'=east}
				[产生原因
					[物理 {} 海流 温度 盐度]
					[化学 {}含高N P等有机物的污水]
					[生物 {}赤潮生物间存在种间增殖竞争]
					]
			\end{forest}
			\item 海洋环境质量评价:根据不同的目的、要求和环境质量标准,按一定评价原则和方法对海洋环境要素的质量进行\textbf{评价和预测},是对海域环境\textbf{规划,管理,及污染防止}的科学依据. 
			
			\textbf{现状评价和影响评价}:影响评价: : 陆源污染物入海后对海洋环境产生的危害程度;海上工程设施和海事活动给海洋环境带来的影响;海洋资源开发过程给海洋环境带来的影响等
			
			\textbf{源头}防止环境污染和生态破坏
			
			\textbf{环境影响评价}:对建设项目和规划实施后可能对环境产生的影响进行\textbf{分析,预测和评估},提出预防或者减轻不良影响的\textbf{对策和措施},进行跟踪监测的\textbf{方法和制度}。 
			
			\item “ 振兴海业,繁荣经济 ”、“ 开发海洋 ”、“ 开发蓝色国土 ”、“ 依法保护和合理开发海洋资源 ”
		\end{enumerate}
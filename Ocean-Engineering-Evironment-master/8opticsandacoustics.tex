\section{Optics and Acoustics in Ocean}
	\subsection{Opitcs}
		\begin{enumerate}
			\item 光在海水中的传播特性:
				\begin{enumerate}
					\item \textbf{反射和折射}: 太阳高度增大时,反射光的能量减小,而折射光的能量增大.
					\item \textbf{吸收和散射}:吸收系数高,光能损失快;可见光中,海水对蓝光和绿光的吸收最少,对红光的吸收最多;海水愈清洁,吸收系数越小.
									 
					
					大洋水中悬浮量少,粒径小,波长较短的蓝光散射能量大,海水颜色呈蓝色;近岸海水悬浮物多,粒径大,黄光散射能量增大,水色多呈黄色、浅蓝等
					\item 光的衰减:长波部分消失较快,短波部分消失较慢.
					透光层euphotic zone,弱光层disphotic zone,无光层aphotic zone.
				\end{enumerate}
			\item 透明度m:实用中一般以直径30cm的白色透明度盘铅直放入水中的最大可见深度来表示
			\item 水色反映海水固有的光学性质.海水及其中悬浮物质及浮游生物等对折射进入水中的太阳光的向上的散射光谱,水色观测用水色标准液进行,根据水色计目测确定.PS海色是海面反射、散射及海水散射等多种光谱组成的颜色.水色$\neq$海色.
			
			水色高透明度大,二者的观测对保证交通运输安全,海上作战及水产养殖生产等有重要作用.
			
			\item 中国近海水色和透明度的分布变化:近海大陆架浅水区域,入 海河流多,浮游物质和悬浮物质含量大,水色低,透明度小。
			自北向南,自近岸到外海,透明度逐渐增大,水色逐渐变高. 
%				\begin{enumerate}
%					\item 渤海的透明度最小,水色低。三湾处水色最低,常黄色,透明度<5m; 中央海区水色呈绿色,透明度可达10m
%					以上。
%					\item 黄海沿岸水色呈黄绿色,透明度约10m;中部海区 水色呈绿色,透明度约15m.
%					\item 东海沿岸水色为绿色,长江口附近呈黄色;受黑潮暖流影响,水色及透明度比黄渤海高,黑潮主流区的水色呈
%					蓝黑色,透明度最大,达 40m 左右
%					\item 南海除珠江口等处,大部分沿岸区呈绿色,透明度约15m ;外海呈深蓝色,透明度40m
%				\end{enumerate}
%			\item 海洋光学的应用
%				 
		\end{enumerate}
	\subsection{Acoustics}
		声波是纵波,是海洋中可进行远距离传播的唯一能量辐射形式.
		
		海水,海面和海底构成一个复杂的声传播空间,存在信号的减弱,延迟和失真.
		
		\begin{enumerate}
			\item 声能损失的原因
			
			\item \begin{forest}
					[sonar
						[driving]
						[passive]
						]
				  \end{forest}
			  \textbf{海洋声学技术}已发展成为海洋高科技中的重要组成部分
			  
			\item 海洋中的声速1450-1540m/s,随\textbf{温度、盐度、深度的增加而增加}
			\item 声波传播符合折射,反射定律.正梯度->向海面弯曲,波导传播,冬季浅海or2000m以下的水层(静压力),不存在海底吸收和反射,所以冬天传得远;负梯度->向海底弯曲,反波导型传播,夏季浅海,海底吸收和散射.
			\item 垂直声速剖面and声速梯度随深度的分布.
				海洋中因某种特定垂直声速剖面的存在,形成能将声能量限制在某一深度范围内而很少向外泄露并使声波传播距离增大的水层,成为声道.水下声道现象:声速极小值所在的深度为声道轴,声波在其可传播很远,各大洋区都存在.
				
				\textbf{表层声道:}海水表层处声速剖面为正梯度时形成的声道轴位于海平面.
				\textbf{深层声道:}深海垂直声速剖面特性所形成的具有深度稳定的声道轴的水层.
		\end{enumerate}
		
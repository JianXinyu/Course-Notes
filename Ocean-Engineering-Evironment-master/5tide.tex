\section{潮汐}
			\subsection{潮汐现象}
			\begin{enumerate}[a)]
				\item 潮汐现象是指海水在天体\textbf{引潮力}作用下所产生的周期性运动
				\item 海面的\textbf{周期性}\textbf{铅直}向涨落————\textbf{潮汐}
				\item 海水在\textbf{水平}方向的流动————\textbf{潮流}
				\item 天体潮 由月球和太阳等天体的引潮力引起的潮汐
				\item 太阴潮 由月球的引潮力引起的潮汐
				\item 太阳潮 有太阳的引潮力引起的潮汐 
			\end{enumerate}
			
			\subsection{潮汐要素定义}
			\begin{enumerate}[a)]
				\item 涨潮 一个潮汐周期内潮位上升的过程
				\item 高潮 潮汐涨落一周期内最高的潮位
				\item 平潮 高潮前后一段时间内海面处在不涨不落的平衡状态
				\item 高潮时 平潮的中间时刻
				\item 落潮 一个潮汐周期内潮位下降的过程
				\item 低潮 潮汐涨落一周期内最低的潮位
				\item 停潮 低潮前后一段时间内海面处在不涨不落的平衡状态
				\item 低潮时 停潮的中间时刻
			\end{enumerate}
			\subsection{潮高计算}
			潮高从潮汐基准面算起;潮差等于相邻高、低潮位之差。
			
			\subsection{引潮力}
			\subsubsection{公转惯性离心力}
			地球绕地月公共质心公转平动的结果,使得地球各质点都受到大小相等、方向相同的公转惯性离心力的作用
			单位质量海水所受的惯性离心力为 $f = GM/D^2$
			\subsubsection{引力}
			$f = GM/L^2$
			\subsubsection{引潮力}
			引潮力指天梯引力和惯性离心力的合力
			其中月球引潮力指的就是地球绕地月公共质心运动产生的惯性离心力与月球引力的合力。
			分解示意图见原版ppt
			\subsubsection{结论}
			\begin{enumerate}[a)]
				\item 引潮力与天体的质量成正比,与天体到地球中心距离的三次方成反比
				\item 月球引潮力是太阳的2.17倍
				\item 海洋潮汐现象主要由月球产生,其实是由太阳产生,其他天体的引潮作用很小。
			\end{enumerate}
				\subsection{平衡潮理论}
				\subsubsection{平衡潮理论假定}
				\begin{enumerate}[a)]
					\item 地球是一个圆球,其表面完全被等深的海水所覆盖,不考虑陆地的存在。
					\item 海水没有粘滞性,也没有惯性,海面能随时与等势面重叠。
					\item 海水不受地转偏向力和摩擦力的作用
					\item 重力与垂向压强梯度力平衡
				\end{enumerate}
				
				\subsubsection{结论}
				引潮力的作用使得潮汐为椭圆形状,其长轴指向月球。
				由于地球自转,地球表面相对于椭球形海面做相对运动,造成某一固定点发生周期性的涨落周期性的潮汐。
				
				\subsection{潮汐的周期性}
				\begin{enumerate}[a)]
					\item 半日周期潮
					月球在赤道上空:一日内有两个高潮和两个低潮,高潮高相等,低潮高也相等。
					\item 月球赤纬不为0的潮汐现象
					在高纬度地区出现正规日潮
					其他一些地区的海面两次高潮的高度不相等,两次涨潮时也不等,形成日不等现象
					赤道仍为正规半日潮
					\item 半月不等现象
					朔望之时,太阳、月球和地球在同一条直线上,高潮最高,低潮最低,潮差最大的潮汐(朔望大潮)
					上下弦时,太阳、月球和地球处于直角位置,两潮相互抵消减弱,潮差最小(方照小潮)
					\item 月不等现象
					月球近地点时潮差较大,远地点时潮差较小。
					\item 年不等现象
					在一年周期中,近日点有一年的变化周期。
					月赤纬有18.61年的变化周期。
				\end{enumerate}
				\subsubsection{平衡潮潮高}
				太阴潮最大潮差54cm
				太阳潮最大潮差24cm
				平衡潮最大潮差78cm
				
				\subsubsection{假想天体和分潮}
				将周期性的潮汐看作是很多不同周期的潮汐叠加而成
				假设每一个分解的潮汐都对应一个天梯——假想天体
				分潮:这些假想天体对海水所引起的潮汐
				
				\subsubsection{潮汐类型的工程计算}
				潮型数 
				
				\subsection{潮汐动力理论}
				\subsubsection{提出}
				缺点 理论与实际不符,没考虑海水的运动与地形的影响。
				拉普拉斯 潮汐动力学理论。从动力学观点出发来研究海水在引潮力作用下产生潮汐的过程
				窄长半封闭的海湾中的潮汐和潮流是驻潮波(波节波腹)
				\subsubsection{潮流}
				引潮力作用下的海水作周期性的水平运动————潮流
				对应潮汐,存在半日周期潮流、日周期潮流、混合潮流
				\subsection{潮汐特征与水位变化}
				\subsubsection{潮位与潮位特征值}
				潮位:受潮汐影响周期性涨落的水位
				平均海平面:消除了各种随机振动和短周期、长周期波动后的理想海面。
				\subsubsection{工程水位}
				设计高、低水位:海洋工程结构物匝正常使用条件下的高(低)水位
				设计高水位:高潮累积频率为百分之10的水位
				设计低水位:低潮累积频率为百分之90的水位
				校核高、低水位:海洋工程结构物在非正常工作条件下的高(低)水位
				利用多年的年最高、最低潮位资料,你和某一极值理论分布曲线,推算多年一遇可能出现的最高(最低)潮位值。
				\subsection{中国近海潮汐与潮流}
				中国近海潮汐主要由太平洋潮波传入,分两支进入中国海区:
				\begin{enumerate}
					\item 进入东海,引起东海,黄海渤海海面发生振动;
					\item 经巴士海峡进入南海,引起南海海面振动。
				\end{enumerate}
				引潮力直接产生的潮汐振动极小,难于存在独立的潮汐系统。
				\subsubsection{潮汐性质}
				\begin{enumerate}
					\item 渤海:大部分为不正规半日潮,小部分为正规全日潮以及不正规全日潮
					\item 东海:主要正规半日潮,部分海域不正规半日潮
					\item 黄海:正规半日潮,小部分不正规半日潮
					\item 南海:绝大部分不正规全日潮 
				\end{enumerate}
				\subsubsection{近海潮流}
				\begin{enumerate}
					\item 渤海:半日潮流为主
					\item 黄海:大部分为正规半日潮流,一部分为不正规全日潮流,流速东大于西
					\item 东海:西部正规半日潮流;东部不正规半日潮流,近岸流速大远岸流速小
					\item 南海 潮流弱
				\end{enumerate}
				\subsubsection{潮差}
				海区中央小,近海岸大
				东海最大,黄海渤海次之,南海最小
				
